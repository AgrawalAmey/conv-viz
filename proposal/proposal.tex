\documentclass{article} % For LaTeX2e
\usepackage{iclr2016_conference,times}
\usepackage{hyperref}
\usepackage{url}


\title{ Project Proposal: Visualizing Convolutional Neural
Networks using Deconvolution }


\author{
Amey Agrawal \& Jaikumar Balani \\
\texttt{\{f2014148, f2014022\}@pilani.bits-pilani.ac.in} \\
}


\newcommand{\fix}{\marginpar{FIX}}
\newcommand{\new}{\marginpar{NEW}}

%\iclrfinalcopy % Uncomment for camera-ready version

\begin{document}


\maketitle

\section{Introduction}

Convolution neural networks have shown groundbreaking results in multiple
computer vision tasks. But CNNs still widely remain a black box model.
Here we attempt at building a framework to visualise a CNN using
deconvolution.

\section{Litrature Review}

Multiple techniques were suggested by \citet{erhan2009visualizing} for visualisation
of CNNs based upon activations of individual units. \citet{simonyan2013deep} established
connection between gradient based methods of CNN visualisation and deconvolution
networks. \citet{zeiler2014visualizing} introduced a way to visualise CNNs using
deconvolution nets.

\section{Deliverables}
The framework would enable the user to visualize individual filter of a CNN on user
input on a graphical interface. We would use Inception v3 checkpoint as our
default network. The front-end UI would be built in JavaScript and HTML, while
the backend would be running on python using tensorflow.

\bibliography{iclr2016_conference}
\bibliographystyle{iclr2016_conference}

\end{document}
