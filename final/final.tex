\documentclass{article} % For LaTeX2e
\usepackage{iclr2016_conference,times}
\usepackage{hyperref}
\usepackage{url}


\title{ Project Report: Visualizing Convolutional Neural
Networks }


\author{
Amey Agrawal \& Jaikumar Balani \\
\texttt{\{f2014148, f2014022\}@pilani.bits-pilani.ac.in} \\
}


\newcommand{\fix}{\marginpar{FIX}}
\newcommand{\new}{\marginpar{NEW}}

%\iclrfinalcopy % Uncomment for camera-ready version

\begin{document}


\maketitle

\section{Abstract}

We are attempting to build a framework to visualise a CNN model,
using multiple methods. Including deconvolution nets
proposed by \citet{zeiler2014visualizing}, activation based method
proposed by \citet{erhan2009visualizing} and t-SNE.


\section{Introduction}

Convolution neural networks have shown groundbreaking results in multiple
computer vision tasks. But CNNs still widely remain a black box model.


\

The framework is built in python with Keras. Keras has been gaining popularity
as deep learning framework due to its simplicity and flexibility. Keras can use
both TensorFlow and Theano backends. We have developed an extensible driver
module with command interactive line interface. It can take any keras model
stored in hdf5 format as input. If the user does not provide an input model VGG16
checkpoint is used by default.

One of the two major methods of CNN visualisation has been implemented, where
we maximise the average of norms of gradients in a given kernel using gradient
ascent. A png of filters in the selected layer is generated as output.

\section{Work Remaining}

The deconvolution based algorithm will be implemented by 14th April.
Deconvolution makes it possible to visualise particular neurones activated in
given filter with the input. We want to develop a system where webcam feed can be used
to see the output of deconvolution net in real time. The real-time nature of the
input makes it challenging. t-SNE visualisations have been proven to be
useful for visualizing high dimensional vector embeddings. For sake of
completeness a t-SNE visualizer would be added to the toolbox to help
visualize the final embeddings generated by the network.


\bibliography{iclr2016_conference}
\bibliographystyle{iclr2016_conference}

\end{document}
